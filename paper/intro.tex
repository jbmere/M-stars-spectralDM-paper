
M-type dwarfs constitute the largest contribution by number to the
Galactic population \cite{2010AJ....139.2679B}. This Galactic
component is extremely important as its properties convey crucial
information about the Galactic structure and evolution. They are known
to harbour terrestrial-sized exoplanets and have recently become a
major target in large scale searches for habitable ones due, amongst
other reasons, to the reduced star-to-planet mass and light
ratios \cite{2015A&A...577A.128A}. Also, because the habitable zone is
significantly closer to the star and thus, the probability to observe
a transit is significantly higher \cite(Shields20161).

These stars span two orders of magnitude in luminosity and almost one
order of magnitude in mass, from 0.075 M$\odot$ to 0.6 M$\odot$. At
0.35 M$\odot$, these stars become fully convective and, given their
low internal pressures, this results in life spans that greatly exceed
the age of the Universe. Although much theoretical work has been
invested in understanding this low mass end of the Main Sequence,
there are still some discrepancies between models and observations
(see e.g. \cite{2013AN....334....4T} for an account of the observed
inflated radii and cooler temperatures with respect to model
predictions).

Given the low atmospheric temperatures, their spectra are heavily
veiled by molecular bands of water (H$_2$O), TiO, VO and H$_2$, to the
extent that defining a true stellar continuum is almost
impossible. The complexity of modelling these molecular bands
(numerous transitions, frequency dependent absortion coefficients,
collisional transitions) and the difficulty in defining a true stellar
continuum prevents the application of standard techniques (common for
F-, G- and K-type stars) to the determination of physical parameters
of M-type stars. 



 who and what. Teff calibrations

\cite{Raj2013}
\cite{Bayo}



 Atlases 
\cite{2013arXiv1306.3709B} % for lambda > 1.1 um :(
\cite{2009ApJS..185..289R} %IRTF library of cool star

\cite{2012ApJ...748...93R} % Metall. and Teff indicators in the K band! :(

 Summary of spectral diagnostics in the IR for M stars


In this work we explore the possibility to define spectral features
for the automated inference of the atmospheric parameters of M type
stars using Machine Learning techniques. In Section \ref{sec:meth} we
describe the methodology used to define and evaluate the spectral
features; in Section \ref{sec:irtf} we apply the methodology described
in Section \ref{sec:meth} in the context of the wavelength coverage
and resolution of the IRTF collection of spectra; we describe the
feature definition results and evaluate them for the task of
predicting physical parameters on the actual observed spectra that
make up the collection. Section \ref{sec:ipac} describes the same
steps in the context of the Dwarf Archives (hereafter IPAC) collection
of spectra. Finally, Section \ref{sec:summary} summarises the main
results and conclusions of the paper.
