
During the preprocessing stage (described in Sect. \ref{meth}) the
spectral resolution of the BT-Settl library was degraded to the IRTF
resolution (R ~ 2000) by convolving with a Gaussian. Then, the spectra
were trimmed to produce valid segments between 8145.92 and
24106.85{\AA}, which is the spectral range common to all M stars in
the IRTF library. Finally, all spectra were divided by the total
integrated flux in this range in order to factor out the stellar
distance.

\subsection{Spectral features for estimation of effective temperatures.}

The application of the GAs to the selection of features for the
prediction of effective temperature from noiseless spectra with the
IRTF wavelength range and resolution results in the features included
in Table~\ref{tab:tab_NC_T}. Features are ordered by the fitness value
(the AIC) and we only consider features that are present in at least 5
sets.


%
% Las tablas de features del GA para Teff están OK
% comprbadas el 2/11/2015 desde 
% http://apii01.etsii.upm.es:8787/files/git/M_prep_IRTF/prep_GA_case01_NT11F2_v1.html
% 
\begin{table}
\begin{center}
\begin{tabular}{rrrrrrr}
  \hline
  $\lambda_1$ & $\lambda_2$ & $\lambda_{cont;1}$ & $\lambda_{cont;2} $ \\ 
  \hline
9225.86  & 9283.94   & 9736.02  & 9793.96 \\
11106.48 & 11193.56  & 13497.81 & 13613.95 \\
13438.08 & 13554.08  & 12006.54 & 12093.56 \\
9135.89  & 9193.91   & 10002.04 & 9999.92 \\
9555.93  & 9614.06   & 12951.62 & 13038.62 \\
9466.08  & 9523.82   & 13137.94 & 13253.96 \\
11196.56 & 11283.24  & 12546.46 & 12633.49 \\
8566.08  & 8624.07   & 13258.32 & 13374.32 \\
8266.11  & 8324.03   & 9856.06  & 9913.91 \\
8235.96  & 8294.04   & 12366.32 & 12453.33 \\
\hline
\end{tabular}
\caption {Features selected by the GA for predicting $T_{eff}$ 
      using BT\_Settl noiseless synthetic
      spectra. } \label{tab:tab_NC_T}
\end{center}
\end{table}

{\bf TBD by Luis: interpret the features.}

When noise is added to the BT-Settl spectra, we obtain 

%The authors have estimated the suggested features when theoretical BT\_Settl 
%is noised with different SNR and following tables \ref{tab:tab_SNR10_T} 
%and \ref{tab:tab_SNR50_T} sumarize the findings.

\begin{table}
\begin{center}
\begin{tabular}{rrrr | rrrr}
  \hline
 \multicolumn{4}{c}{SNR = 10} &  \multicolumn{4}{c}{SNR=50} \\
  \hline
$\lambda_1$ & $\lambda_2$ & $\lambda_{cont;1}$ & $\lambda_{cont;2} $ & $\lambda_1$ & $\lambda_2$ & $\lambda_{cont;1}$ & $\lambda_{cont;2} $ \\ 
  \hline
8235.96  & 8294.04   & 12681.62 & 12768.68   &  8145.92 & 8204.03   & 12636.48 & 12723.57 \\   
8505.89  & 8563.93   & 13378.12 & 13494.13   &  8895.95 & 8953.95   & 11331.57 & 11418.65 \\     
9376.07  & 9433.92   & 12951.62 & 13038.62   &  8176.03 & 8234.13   & 10611.36 & 10698.46 \\      
8145.92  & 8204.03   & 12366.32 & 12453.33   &  13438.08 & 13554.08 & 12546.46 & 12633.49 \\     
9195.86  & 9253.93   & 9135.89 & 9193.92     &  8235.96 & 8294.04   & 11961.44 & 12048.54 \\      
9585.95  & 9644.12   & 10002.04 & 9999.92    &  9376.07 & 9433.92   & 10002.04 & 9999.92  \\   
8385.99  & 8443.94   & 11826.48 & 11913.28   &  9406.09 & 9463.96   & 13258.32 & 13374.32 \\    
9135.89  & 9193.92   & 9225.86 & 9283.94     &  9346.13 & 9403.92   & 13086.46 & 13194.09 \\   
13618.20 & 13734.15  & 11376.63 & 11463.51   &  11106.48 & 11193.56 & 13438.08 & 13554.08 \\    
9105.87  & 9163.91   & 8865.98 & 8923.94     &  9255.86 & 9314.01   & 8865.98  & 8923.94  \\    
\hline
\end{tabular}
\caption {Recommended features and Continuum bandpass for predicting $ T_{eff} $ 
      by using BT\_Settl with SNR= $ 10 $ and 50.} \label{tab:tab_SNR1050_T} 
\end{center}
\end{table}

Tables \ref{tab:tab_NC_T} and \ref{tab:tab_SNR1050_t} show a very wide
variety of features with very few repetitions. Only spectral features
4, 5, 6, and 9 in the SNR=50 experiment are found too in the
SNR=$\infty$ and SNR=10 feature sets (albeit with different continuum
definitions). This reinforces the impression that the information
useful for the estimation of the effective temperatures is spread over
the entire IRTF spectrum.

A closer look at features 4, 5, 6, and 9 

As a reference, Table~\ref{tab:tab_cesetti} lists the features found
by \cite{2013A&A...549A.129C} using sensitivity maps. 

\begin{table}
\begin{center}
\begin{tabular}{rrrrrrrr}
  \hline
Index & Element & Signal\_from & Signal\_To & Cont1\_From & Cont1\_To & Cont2\_From & Cont2\_To \\ 
  \hline
  Pa1 & H~{\sc i}   & 8461 & 8474 & 8474 & 8484 & 8563 & 8577 \\ 
  Ca1 & Ca~{\sc ii} & 8484 & 8513 & 8474 & 8484 & 8563 & 8577 \\ 
  Ca2 & Ca~{\sc ii} & 8522 & 8562  & 8474 & 8484 & 8563 & 8577 \\ 
  Pa2 & H~{\sc i}   & 8577 & 8619 & 8563 & 8577 & 8619 & 8642 \\ 
  Ca3 & Ca~{\sc ii} & 8642 & 8682 & 8619 & 8642 & 8700 & 8725 \\ 
  Pa3 & H~{\sc i}   & 8730 & 8772 & 8700 & 8725 & 8776 & 8792 \\ 
  Mg  & Mg~{\sc i}  & 8802 & 8811 & 8776 & 8792 & 8815 & 8850 \\ 
  Pa4 & H~{\sc i}   & 8850 & 8890 & 8815 & 8850 & 8890 & 8900 \\ 
  Pa5 & H~{\sc i}   & 9000 & 9030 & 8983 & 8998 & 9040 & 9050 \\
  FeClTi & Fe~{\sc i}, Cl~{\sc i}, Ti~{\sc i} &9080 & 9100 & 9040 & 9050 & 9125 & 9135 \\
  Pa6 & H~{\sc i}   &9217 & 9255 & 9152 & 9165 & 9265 & 9275 \\
Fe1         & Fe{~\sc i}     & 1.9297 & 1.9327~& 1.9220 & 1.9260 & 2.0030 & 2.0100 \\
Br$\delta$  & H{~\sc i} (n=4)& 1.9425 & 1.9470~& 1.9220 & 1.9260 & 2.0030 & 2.0100 \\
Ca1         & Ca{~\sc i}     & 1.9500 & 1.9526~& 1.9220 & 1.9260 & 2.0030 & 2.0100 \\
Fe23        & Fe{~\sc i}     & 1.9583 & 1.9656~& 1.9220 & 1.9260 & 2.0030 & 2.0100 \\
Si          & Si{~\sc i}     & 1.9708 & 1.9748~& 1.9220 & 1.9260 & 2.0030 & 2.0100 \\
Ca2         & Ca{~\sc i}     & 1.9769 & 1.9795~& 1.9220 & 1.9260 & 2.0030 & 2.0100 \\
Ca3         & Ca{~\sc i}     & 1.9847 & 1.9881~& 1.9220 & 1.9260 & 2.0030 & 2.0100 \\
Ca4         & Ca{~\sc i}     & 1.9917 & 1.9943~& 1.9220 & 1.9260 & 2.0030 & 2.0100 \\
Mg1         & Mg{~\sc i}     & 2.1040 & 2.1110~& 2.1000 & 2.1040 & 2.1110 & 2.1150 \\
Br$\gamma$  & H{~\sc i} (n=4)& 2.1639 & 2.1686~& 2.0907 & 2.0951 & 2.2873 & 2.2900 \\
Na$_{\rm d}$& Na{~\sc i}     & 2.2000 & 2.2140~& 2.1934 & 2.1996  & 2.2150 & 2.2190   \\
FeA         & Fe{~\sc i}     & 2.2250 & 2.2299~& 2.2133 & 2.2176 & 2.2437 & 2.2479 \\
FeB         & Fe{~\sc i}     & 2.2368 & 2.2414~& 2.2133 & 2.2176 & 2.2437 & 2.2479 \\
Ca$_{\rm d}$& Ca{~\sc i}     & 2.2594 & 2.2700~& 2.2516 & 2.2590  & 2.2716 & 2.2888   \\
Mg2         & Mg{~\sc i}     & 2.2795 & 2.2845~& 2.2700 & 2.2720 & 2.2850 & 2.2874 \\
$^{12}$CO   & $^{12}$CO(2,0) & 2.2910 & 2.3070~& 2.2516 & 2.2590   & 2.2716 & 2.2888   \\
\hline
\end{tabular}
\caption {Recommended features and continuum bandpasses recommended in 
   \cite{2013A&A...549A.129C} as relevant for the estimation of the
   effective temperature in bands I and K.} \label{tab:tab_cesetti}
\end{center}
\end{table}


For gravity (in the form of $\log(g)$) estimation, the GA search
procedure produces the features presented in
Tables \ref{tab:tab_SNRoo_G} and \ref{tab:tab_SNR1050_G} for the pure
synthetic signal and signal-to-noise ratios of 10 and 50,
respectively.

%
% Las tablas de features del GA para Teff están OK
% comprbadas el 2/11/2015 desde 
% http://apii01.etsii.upm.es:8787/files/git/M_prep_IRTF/prep_GA_case01_NG11F2_v1.html
% 
\begin{table}
\begin{center}
\begin{tabular}{rrrrrrr}
  \hline
  $\lambda_1$ & $\lambda_2$ & $\lambda_{cont;1}$ & $\lambda_{cont;2} $ \\ 
  \hline
     10245.88 & 10304.02 &	11241.29 & 11328.54 \\
     8415.91  & 8473.96  &	11511.51 & 11598.51\\
     12906.56 & 12993.61 &	13041.48 & 13133.82\\
     8716.00  & 8773.99  &	10425.90 & 10484.13\\
     8805.93  & 8863.97  &	12816.72 & 12903.73\\
     10126.02 & 10183.93 &	13086.46 & 13194.09\\
     8176.03  & 8234.13  &	10971.57 & 11058.46\\
     8626.02  & 8683.99  &	10746.43 & 10833.57\\
     8536.03  & 8594.06  &	10215.95 & 10274.10\\
     12951.62 & 13038.62 &	11196.56 & 11283.24 \\

\hline
\end{tabular}
\caption {Recommended features and continuum bandpasses for predicting $\log(g)$ 
      obtained using noiseless BT\_Settl spectra.} \label{tab:tab_SNRoo_G}
\end{center}
\end{table}

\begin{table}
\begin{center}
\begin{tabular}{rrrr | rrrr}
  \hline
 \multicolumn{4}{c}{SNR = 10} &  \multicolumn{4}{c}{SNR=50} \\
  \hline
$\lambda_1$ & $\lambda_2$ & $\lambda_{cont;1}$ & $\lambda_{cont;2} $ & $\lambda_1$ & $\lambda_2$ & $\lambda_{cont;1}$ & $\lambda_{cont;2} $ \\ 
  \hline
     8176.03  & 8234.13  &	9165.87  & 9223.91  &  11151.63 & 11238.46 &      13086.46 & 13194.09 \\
     10485.99 & 10563.41 &	10002.04 & 9999.92  &  8385.99  & 8443.94  &      13618.20 & 13734.14 \\
     8656.09  & 8714.047 &      10926.46 & 11013.60 &  8176.03  & 8234.13  &      11241.29 & 11328.54 \\
     9525.89  & 9584.059 &	10002.04 & 9999.92  &  8536.03  & 8594.06  &      13041.48 & 13133.82 \\ 
     8205.98  & 8263.967 &	13041.48 & 13133.82 &  12771.70 & 12858.73 &      10306.03 & 10363.88 \\
     10275.97 & 10333.96 &	11376.63 & 11463.51 &  13378.12 & 13494.13 &      10002.04 & 9999.92  \\
     10306.03 & 10363.88 &	11151.63 & 11238.46 &  8626.02  & 8683.99  &      10926.46 & 11013.60 \\
     9165.87  & 9223.91  &	8385.99  & 8443.94  &  9826.05  & 9883.91  &      10006.07 & 10064.01 \\
     9645.82  & 9704.16  &	13137.94 & 13253.96 &  10521.56 & 10608.46 &      11736.71 & 11823.49 \\
     8326.00  & 8383.94  &	12726.69 & 12813.71 &  8205.98  & 8263.96  &      9796.09  & 9853.94  \\ 
   \hline
\end{tabular}
\caption {Recommended features and continuum bandpasses for predicting $log(g)$ 
      obtained using BT\_Settl with SNR= $10$ and
      50.} \label{tab:tab_SNR1050_G}
\end{center}
\end{table}


Finally, the best features found by the GA for metallicity estimation
are listed in Table~\ref{tab:tab_SNRoo_M} for the noiseless BT-Settl
spectra, and in Table~\ref{tab:tab_SNR1050_M} for signal-to-noise
ratios equal to 10 and 50.

%
% Las tablas de features del GA para Teff están OK
% comprbadas el 2/11/2015 desde 
% http://apii01.etsii.upm.es:8787/files/git/M_prep_IRTF/
% 
\begin{table}
\begin{center}
\begin{tabular}{rrrrrrr}
  \hline
  $\lambda_1$ & $\lambda_2$ & $\lambda_{cont;1}$ & $\lambda_{cont;2} $ \\ 
  \hline
     12096.68 & 12183.66  & 12051.50 & 12096.68 \\
     9525.89 & 9584.05 	  & 12321.33 & 12408.32 \\
     8205.98 & 8263.96 	  & 10126.02 & 10183.93 \\
     8566.08 & 8624.07 	  & 12276.52 & 12363.34 \\
     11196.56 & 11283.24  & 11151.63 & 11196.56 \\
     11151.639 & 11238.46 & 11466.35 & 11553.33 \\
     9555.93 & 9614.06 	  & 8205.98  & 8263.96 \\
     11016.62 & 11103.37  & 10791.44 & 10878.40 \\
     9766.16 & 9823.94 	  & 12681.62 & 12768.68 \\
     9942.14 & 9999.92   & 9555.93  & 9614.06 \\
\hline
\end{tabular}
\caption {Feature and Continuum bandpasses selected for predicting $Metallicity$ 
      using noiseless BT\_Settl spectra.} \label{tab:tab_SNRoo_M}
\end{center}
\end{table}

When signal-to-noise ratios equal to 10 and 50 are considered, the
GA finds the selected features listed in Table \ref{tab:tab_SNR1050_M}.


\begin{table}
\begin{center}
\begin{tabular}{rrrr | rrrr}
  \hline
 \multicolumn{4}{c}{SNR = 10} &  \multicolumn{4}{c}{SNR=50} \\
  \hline
$\lambda_1$ & $\lambda_2$ & $\lambda_{cont;1}$ & $\lambda_{cont;2} $ & $\lambda_1$ & $\lambda_2$ & $\lambda_{cont;1}$ & $\lambda_{cont;2} $ \\ 
  \hline

8235.96  & 8294.04  &	11331.57 & 11418.65 & 9255.86 & 9314.01   &     13197.94 & 13313.92  \\
9376.07  & 9433.92  &	10566.33 & 10653.62 & 8385.99 & 8443.94   &     9376.07  & 9433.92    \\
10306.03 & 10363.88 &	 9942.14 & 9999.92  & 8716.00 & 8773.99   &     9585.95  & 9644.12    \\
11286.42 & 11373.45 &	11241.29 & 11286.42 & 8235.96 & 8294.04   &     13086.46 & 13194.09  \\
9676.00  & 9734.02  &	13086.46 & 13194.09 & 9676.00 & 9734.02   &     10791.44 & 10878.40  \\
8775.95  & 8833.94  &	8415.91  & 8473.96  & 8415.91 & 8473.96   &     12411.34 & 12498.41  \\
12411.34 & 12498.41 &	10245.88 & 10304.02 & 8446.03 & 8503.94   &     9406.09  & 9463.96    \\
8476.01  & 8534.03  &	12276.52 & 12363.34 & 8205.98 & 8263.96   &     8955.88  & 9013.95    \\
12636.48 & 12723.57 &	12051.50 & 12138.72 & 8985.93 & 9043.98   &     12186.62 & 12273.48  \\
8415.91  & 8473.96  &	13618.20 & 13734.14 & 9015.98 & 9073.98   &     11241.29 & 11328.54  \\

\hline
\end{tabular}
\caption {Feature and Continuum bandpasses selected for predicting $Metallicity$ 
      using noiseless BT\_Settl spectra with signal-to-noise ratios
      equal to $10$ and 50.} \label{tab:tab_SNR1050_G}
\end{center}
\end{table}

