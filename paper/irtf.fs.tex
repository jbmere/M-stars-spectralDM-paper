
During the preprocessing stage (described in Sect. \ref{sec:meth}) the
spectral resolution of the BT-Settl library was degraded to the IRTF
resolution (R ~ 2000) by convolving with a Gaussian. Then, the spectra
were trimmed to produce valid segments between 8145.92 and
24106.85{\AA}, which is the spectral range common to all M stars in
the IRTF library. Finally, all spectra were divided by the total
integrated flux in this range in order to factor out the stellar
distance.

\subsubsection{Spectral features for the estimation of effective temperatures.}

The application of the GAs to the selection of features for the
prediction of effective temperature from noiseless spectra with the
IRTF wavelength range and resolution results in the features included
in Table~\ref{tab:irtf-teff-noiseless}. Features are ordered by the
fitness value (the AIC) and we only consider features that are present
in at least 5 sets.

When noise is added to the BT-Settl spectra, we obtain the features
included in Table \ref{tab:irtf-teff-noisy}.

%The authors have estimated the suggested features when theoretical BT\_Settl 
%is noised with different SNR and following tables \ref{tab:tab_SNR10_T} 
%and \ref{tab:tab_SNR50_T} sumarize the findings.

Tables \ref{tab:irtf-teff-noiseless} and \ref{tab:irtf-teff-noisy}
show a very wide variety of features with very few repetitions. Only
spectral features 4, 5, 6, and 9 in the SNR=50 experiment are found
too in the SNR=$\infty$ and SNR=10 feature sets (albeit with different
continuum definitions). This reinforces the impression that the
information useful for the estimation of the effective temperatures is
spread over the entire IRTF spectrum.

As a reference, Table~\ref{tab:irtf-cesetti} lists the features found
by \cite{cesetti} using sensitivity maps.

For gravity estimation (on a logarithmic scale), the GA search
procedure produces the features presented in
Tables \ref{tab:irtf-logg-noiseless} and \ref{tab:irtf-logg-noisy} for
the noiseless signal and signal-to-noise ratios of 10 and 50,
respectively.

Finally, the best features found by the GA for metallicity estimation
are listed in Table~\ref{tab:irtf-met-noiseless} for the noiseless BT-Settl
spectra, and in Table~\ref{tab:irtf-met-noisy} for signal-to-noise
ratios equal to 10 and 50.

Figures \ref{fig:irtf-teff-inf}-\ref{fig:irtf-met-50} shows
graphically the band limits listed in
Tables \ref{tab:irtf-teff-noiseless}--\ref{tab:irtf-met-noisy} on a
collection of spectra from the BT-Settl library.


