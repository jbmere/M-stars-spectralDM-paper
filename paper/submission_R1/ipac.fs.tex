
As for the IRTF spectra, the spectral resolution of the BT-Settl
library was degraded to match the average resolution of IPAC spectra
in the Dwarf
Archives\footnote{http://spider.ipac.caltech.edu/staff/davy/ARCHIVE/index.shtml}. Then,
the spectra were trimmed to produce segments in the spectral range
common to all spectra of M stars in the archive, to avoid missing data
in the input variables. Finally, all spectra were divided by the total
integrated flux in this range in order to factor out the stellar
distance.

There is little hope {\it a priori} for reasonable accuracies with
regression models that predict the surface gravity and metallicity
from such wavelength-limited, low/intermediate resolution
spectra. Anyhow, we provide the results obtained applying the same
methodology as in Section \ref{sec:irtf} (and described in
Section \ref{sec:meth}) to show the limitations.

The application of the GA to the selection of features for the
prediction of effective temperature from noiseless and noisy spectra within the
DA wavelength range and resolution, results in the features included
in Table~\ref{tab:ipac-teff-noisy}. Features are ordered by the
fitness value (according to the AIC criterion presented in Eq.~\ref{eq:AIC}), 
and we only consider features that are present
in at least 5 sets. 
%For the noisy spectra of SNR=10 and 50 we select
%the spectral features listed in Table~\ref{tab:ipac-teff-noisy}.

Table   %\ref{tab:ipac-logg-noiseless} and 
\ref{tab:ipac-logg-noisy}
shows the spectral features selected by the GA for noiseless BT-Settl
spectra and the same spectra with SNR=10 and 50, respectively.

Finally, the best features found by the GA for the estimation of the
metallicity are listed in 
% Table~\ref{tab:ipac-met-noiseless} for the
% noiseless BT-Settl spectra, and in 
Table~\ref{tab:ipac-met-noisy} for
signal-to-noise ratios equal to $\infty , 10 $ and $ 50 $.

