
M-type dwarfs constitute the largest contribution by number to the
Galactic population \citep{2010AJ....139.2679B}. This Galactic
component is very important as its properties convey crucial
information about the Galactic structure and
evolution \citep{2013A&A...556A.110B}. They are known to harbour
super-Earth exoplanets \citep{2013A&A...556A.110B} and have recently
become a major target in large-scale searches for habitable ones due,
amongst other reasons, to the reduced star-to-planet mass and light
ratios \citep{2015A&A...577A.128A}. In addition, the probability to observe a
transit is significantly higher because the habitable zone is
significantly closer to the star \citep{Shields20161}.

These stars span two orders of magnitude in luminosity and almost one
order of magnitude in mass, from 0.075 M$\odot$ to 0.6 M$\odot$. At
0.35 M$\odot$, these stars become fully convective and, given their
low internal pressures, this results in life spans that greatly exceed
the age of the Universe. Although much theoretical work has been
invested in understanding this low mass end of the Main
Sequence \citep{2008ApJ...676.1262B}, there are still some
discrepancies between models and observations \citep[see e.g.][for an
account of the observed inflated radii and cooler temperatures with
respect to model predictions]{2013AN....334....4T}.

Molecular bands of water (H$_2$O), TiO, VO,
heavily veil their spectra and H$_2$, to the extent that defining a true stellar
continuum is difficult. The complexity of modelling
these molecular bands (numerous transitions, frequency-dependent
absorption coefficients, collisional transitions) and the difficulty in
defining a true stellar continuum prevents the application of standard
techniques (common for F-, G- and K-type stars) to the determination
of physical parameters of M-type stars.

% The state-of-the-art
% Summary of spectral diagnostics in the IR for M stars
Given the prevalence of these late-type stars, it has become
increasingly important to be able to estimate their atmospheric
physical parameters with reproducible methods that provide homogeneous
values for large samples of spectra. \cite{2012ApJ...748...93R}
proposed the H2O-K2, Na~{\sc I} and Ca~{\sc I} spectral indices to
estimate spectral type, effective temperature ($T_{\rm eff}$) and
metallicity from K band spectra between 1.0 and 2.4 $\mu$m at a
resolution R$\approx$2700. They predicted spectral types with a quoted
accuracy of 0.6 sub-types, and metallicities with a root mean square
error (RMSE) of 0.1 dex (0.14 dex for the Iron abundance). The
definition of the line/band spectral regions and pseudo-continuum is
justified in terms of contiguity and the avoidance of other atomic
features. 

\cite{2014A&A...568A.121N} on the contrary concentrated in
high resolution (R$\approx$ 115000) spectra in the optical range.
They proposed a method based on a linear least squares fit of the
equivalent widths (EWs) of 4104 lines in the 530-690 nm spectral range
to effective temperatures and metallicities derived from the scales
by \cite{2012A&A...538A..25N} and \cite{2008MNRAS.389..585C}.  They
show RMSE of 0.12 dex for the metallicity and 293 K for the effective
temperature. \cite{2015ApJ...800...85N} developed a calibration of
effective temperatures, radii and bolometric luminosities with Mg and
Al spectral features measured in low-resolution near-infrared spectra
(from the SpeX instrument on IRTF, the same instrument used and
described in Section \ref{sec:irtf} of this work).  They quoted 
residual standard deviations of the $T_{\rm eff}$ fit of 73
K.

% I'm including here the reference to the requested paper Mann2013 ... JOM 25/11/2017
\textbf{
\cite{2013AJ....145...52M} look to determine metallicities of 
late K and M dwarfs from moderate resolution (1300 < R < 2000) 
observing visible and infrared spectra. The authors propose to 
empirically identify the most convenient features relevant 
to already known metalicities and they generalize from those features.
}
% End of the citation for that work.
\cite{Mann2015} calculate effective temperatures using the
classical $\chi^2$ minimization of the observed optical spectra with
respect to the CIFIST2011 library of BT-Settl synthetic spectra.
Uncertaintites of temperatures are estimated around 60 K.
Metallicities are estimated from the EWs of atomic spectral lines in
near-infrared spectra (again SpeX) using calibrations obtained from
wide binaries with FGK primaries. The estimated errors in metallicity
are 0.08 dex.

The previous summary shows a lack of estimates for the surface gravity
and a variety of methodologies for the estimation of temperatures and
metallicities. In this work we are mainly concerned with estimating
atmospheric physical parameters using spectral features and libraries
of synthetic spectra. Our aim is to identify the best spectral
features for estimating $T_{\rm eff}$, $\log(g)$ and
$[M/H]$. \cite{cesetti} proposed sensitivity maps (the derivative
of the monochromatic fluxes with respect to the atmospheric physical
parameters) to rank spectral features. In this work we explore the
validity of the features proposed by \cite{cesetti} and propose and
evaluate new features using standard machine learning (ML) techniques. In
Section \ref{sec:meth} we describe the methodology used to define and
evaluate the spectral features; in Section \ref{sec:irtf} we apply the
methodology described in Section \ref{sec:meth} in the context of the
wavelength coverage and resolution of the IRTF collection of spectra;
we describe the feature definition results and evaluate them for the
task of predicting physical parameters on the actual observed spectra
that make up the collection. Section \ref{sec:ipac} describes the same
steps in the context of the Dwarf Archives (hereafter DA) collection
of spectra. Finally, Section \ref{sec:summary} summarises the main
results and conclusions of the paper.
