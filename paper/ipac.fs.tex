
As for the IRTF spectra, the spectral resolution of the BT-Settl
library was degraded to match the average resolution of IPAC spectra
in the Dwarf
Archives\footnote{http://spider.ipac.caltech.edu/staff/davy/ARCHIVE/index.shtml}. {\bf
What is the average resolution?}. Then, the spectra were trimmed to
produce valid segments between *** and *** {\AA}, which is the
spectral range common to all M stars in the archive. Finally, all
spectra were divided by the total integrated flux in this range in
order to factor out the stellar distance.

There is little hope {\it a priori} for reasonable accuracies with
regression models that predict the surface gravity and metallicity
from such wavelength-limited, low/intermediate resolution
spectra. Anyhow, we provide the results obtained applying the same
methodology as in Section \ref{irtf} to show the limitations.

\subsubsection{Spectral features for the estimation of effective temperatures.}

The application of the GA to the selection of features for the
prediction of effective temperature from noiseless spectra within the
IPAC wavelength range and resolution, results in the features included
in Table~\ref{tab:tab_NC_T}. Features are ordered by the fitness value
(the AIC){\bf and we only consider features that are present in at least 5
sets}.

{\bf TBD by Luis: interpret the features.}

When noise is added to the BT-Settl spectra, we obtain the following
features depending on the SNR of the spectra:

Tables \ref{tab:tab_SNRoo_G} and \ref{tab:tab_SNR1050_G} show the
spectral features selected by the GA for noiseless BT-Settl spectra
and the same spectra with SNR=10 and 50, respectively.

Finally, the best features found by the GA for the estimation of the
metallicity are listed in Table~\ref{tab:tab_SNRoo_M} for the
noiseless BT-Settl spectra, and in Table~\ref{tab:tab_SNR1050_M} for
signal-to-noise ratios equal to 10 and 50.

