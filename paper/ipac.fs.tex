
During the preprocessing stage (and in a similar procedure as used in
the case of the IRTF spectra) the spectral resolution of the BT-Settl
library was degraded to match the average resolution of spectra in the
Dwarf Archives by convolving with a Gaussian. Then, the spectra were
trimmed to produce valid segments between *** and *** {\AA}, which is
the spectral range common to all M stars in the archive. Finally, all
spectra were divided by the total integrated flux in this range in
order to factor out the stellar distance.

\subsection{Spectral features for estimation of effective temperatures.}

The application of the GAs to the selection of features for the
prediction of effective temperature from noiseless spectra with the
IRTF wavelength range and resolution results in the features included
in Table~\ref{tab:tab_NC_T}. Features are ordered by the fitness value
(the AIC) and we only consider features that are present in at least 5
sets.

\begin{table}
\begin{center}
\begin{tabular}{rrrr}
  \hline
  $\lambda_1$ & $\lambda_2$ & $\lambda_{cont;1}$ & $\lambda_{cont;2} $ \\ 
  \hline
7062 & 7094.4 &	7314 & 7346.4 \\
7116 & 7148.4 &	7782 & 7814.4 \\
7134 & 7166.4 &	7872 & 7904.4 \\
6900 & 6932.4 &	7764 & 7796.4 \\
7170 & 7202.4 &	7890 & 7922.4 \\
7080 & 7112.4 &	7926 & 7958.4 \\
7188 & 7220.4 &	7548 & 7580.4 \\
7800 & 7832.4 &	7962 & 7994.4 \\
6990 & 7022.4 &	7008 & 7040.4 \\
7026 & 7058.4 &	6990 & 7022.4 \\
\hline
\end{tabular}
\caption {Features selected by the GA for predicting $T_{eff}$ 
      using BT\_Settl noiseless synthetic
      spectra. } \label{tab:tab_NC_T}
\end{center}
\end{table}

{\bf TBD by Luis: interpret the features.}

When noise is added to the BT-Settl spectra, we obtain 

%The authors have estimated the suggested features when theoretical BT\_Settl 
%is noised with different SNR and following tables \ref{tab:tab_SNR10_T} 
%and \ref{tab:tab_SNR50_T} sumarize the findings.

\begin{table}
\begin{center}
\begin{tabular}{rrrr | rrrr}
  \hline
 \multicolumn{4}{c}{SNR = 10} &  \multicolumn{4}{c}{SNR=50} \\
  \hline
$\lambda_1$ & $\lambda_2$ & $\lambda_{cont;1}$ & $\lambda_{cont;2} $ & $\lambda_1$ & $\lambda_2$ & $\lambda_{cont;1}$ & $\lambda_{cont;2} $ \\ 
  \hline
7692 & 7724.4 	6936 & 6968.4  & 7062 & 7094.4   7296 & 7328.4 \\
6990 & 7022.4 	7998 & 8030.4  & 7026 & 7058.4   7044 & 7076.4 \\
6900 & 6932.4 	7548 & 7580.4  & 7080 & 7112.4   7926 & 7958.4 \\
7854 & 7886.4 	7710 & 7742.4  & 6900 & 6932.4   7548 & 7580.4 \\
7116 & 7148.4 	7908 & 7940.4  & 7134 & 7166.4   7836 & 7868.4 \\
7278 & 7310.4 	7926 & 7958.4  & 7296 & 7328.4   7962 & 7994.4 \\
7152 & 7184.4 	7746 & 7778.4  & 6936 & 6968.4   7728 & 7760.4 \\
7134 & 7166.4 	7764 & 7796.4  & 6972 & 7004.4   6900 & 6932.4 \\
6918 & 6950.4 	6900 & 6932.4  & 6990 & 7022.4   7944 & 7976.4 \\
7224 & 7256.4 	7962 & 7994.4  & 6918 & 6950.4   7782 & 7814.4 \\
\hline
\end{tabular}
\caption {Recommended features and Continuum bandpass for predicting $ T_{eff} $ 
      by using BT\_Settl with SNR= $ 10 $ and 50.} \label{tab:tab_SNR1050_T} 
\end{center}
\end{table}

For gravity (in the form of $\log(g)$) estimation, the GA search
procedure produces the features presented in
Tables \ref{tab:tab_SNRoo_G} and \ref{tab:tab_SNR1050_G} for the pure
synthetic signal and signal-to-noise ratios of 10 and 50,
respectively.

\begin{table}
\begin{center}
\begin{tabular}{rrrr}
  \hline
  $\lambda_1$ & $\lambda_2$ & $\lambda_{cont;1}$ & $\lambda_{cont;2} $ \\ 
  \hline

7134 & 7166.4 &	7044 & 7076.4 \\
6954 & 6986.4 &	7152 & 7184.4 \\
7512 & 7544.4 &	7890 & 7922.4 \\
7062 & 7094.4 &	7224 & 7256.4 \\
6936 & 6968.4 &	7854 & 7886.4 \\
6900 & 6932.4 &	7746 & 7778.4 \\
6918 & 6950.4 &	7800 & 7832.4 \\
7008 & 7040.4 &	7134 & 7166.4 \\
7872 & 7904.4 &	7008 & 7040.4 \\
7962 & 7994.4 &	7980 & 8012.4 \\

\hline
\end{tabular}
\caption {Recommended features and continuum bandpasses for predicting $\log(g)$ 
      obtained using noiseless BT\_Settl spectra.} \label{tab:tab_SNRoo_G}
\end{center}
\end{table}

\begin{table}
\begin{center}
\begin{tabular}{rrrr | rrrr}
  \hline
 \multicolumn{4}{c}{SNR = 10} &  \multicolumn{4}{c}{SNR=50} \\
  \hline
$\lambda_1$ & $\lambda_2$ & $\lambda_{cont;1}$ & $\lambda_{cont;2} $ & $\lambda_1$ & $\lambda_2$ & $\lambda_{cont;1}$ & $\lambda_{cont;2} $ \\ 
  \hline

6990 & 7022.4 &	6918 & 6950.4 & 6918 & 6950.4 & 6936 & 6968.4  \\
6900 & 6932.4 &	7278 & 7310.4 & 6936 & 6968.4 & 7836 & 7868.4  \\
7062 & 7094.4 &	7242 & 7274.4 & 7656 & 7688.4 & 7890 & 7922.4  \\
7692 & 7724.4 &	7008 & 7040.4 & 6900 & 6932.4 & 7872 & 7904.4  \\
7656 & 7688.4 &	7998 & 8030.4 & 7008 & 7040.4 & 7044 & 7076.4  \\
6936 & 6968.4 &	7836 & 7868.4 & 7512 & 7544.4 & 7656 & 7688.4  \\
7206 & 7238.4 &	7062 & 7094.4 & 7440 & 7472.4 & 7332 & 7364.4  \\
7512 & 7544.4 &	7926 & 7958.4 & 7800 & 7832.4 & 7692 & 7724.4  \\
7764 & 7796.4 &	7710 & 7742.4 & 7404 & 7436.4 & 7548 & 7580.4  \\
7404 & 7436.4 &	7548 & 7580.4 & 7080 & 7112.4 & 7152 & 7184.4  \\
   \hline
\end{tabular}
\caption {Recommended features and continuum bandpasses for predicting $log(g)$ 
      obtained using BT\_Settl with SNR= $10$ and
      50.} \label{tab:tab_SNR1050_G}
\end{center}
\end{table}


Finally, the best features found by the GA for metallicity estimation
are listed in Table~\ref{tab:tab_SNRoo_M} for the noiseless BT-Settl
spectra, and in Table~\ref{tab:tab_SNR1050_M} for signal-to-noise
ratios equal to 10 and 50.

\begin{table}
\begin{center}
\begin{tabular}{rrrr}
  \hline
  $\lambda_1$ & $\lambda_2$ & $\lambda_{cont;1}$ & $\lambda_{cont;2} $ \\ 
  \hline
7188 & 7220.4 &	7854 & 7886.4 \\ 
7080 & 7112.4 &	7926 & 7958.4 \\
7116 & 7148.4 &	7098 & 7130.4 \\
7422 & 7454.4 &	7836 & 7868.4 \\
7350 & 7382.4 &	7998 & 8030.4 \\
7224 & 7256.4 &	7818 & 7850.4 \\
7710 & 7742.4 &	7062 & 7094.4 \\
7476 & 7508.4 &	7944 & 7976.4 \\
7134 & 7166.4 &	7584 & 7616.4 \\
7836 & 7868.4 &	7278 & 7310.4 \\
\hline
\end{tabular}
\caption {Feature and Continuum bandpasses selected for predicting $Metallicity$ 
      using noiseless BT\_Settl spectra.} \label{tab:tab_SNRoo_M}
\end{center}
\end{table}

When signal-to-noise ratios equal to 10 and 50 are considered, the
GA finds the selected features listed in Table \ref{tab:tab_SNR1050_M}.


\begin{table}
\begin{center}
\begin{tabular}{rrrr | rrrr}
  \hline
 \multicolumn{4}{c}{SNR = 10} &  \multicolumn{4}{c}{SNR=50} \\
  \hline
$\lambda_1$ & $\lambda_2$ & $\lambda_{cont;1}$ & $\lambda_{cont;2} $ & $\lambda_1$ & $\lambda_2$ & $\lambda_{cont;1}$ & $\lambda_{cont;2} $ \\ 
  \hline
7692 & 7724.4 &	7026 & 7058.4  &  7098 & 7130.4 & 7926 & 7958.4 \\
6900 & 6932.4 &	7008 & 7040.4  &  7188 & 7220.4 & 7962 & 7994.4  \\
7350 & 7382.4 &	7908 & 7940.4  &  7368 & 7400.4 & 7980 & 8012.4  \\
6918 & 6950.4 &	6900 & 6932.4  &  7116 & 7148.4 & 7872 & 7904.4  \\
7098 & 7130.4 &	7314 & 7346.4  &  7062 & 7094.4 & 7206 & 7238.4  \\
7440 & 7472.4 &	7872 & 7904.4  &  7584 & 7616.4 & 7170 & 7202.4  \\
7134 & 7166.4 &	7962 & 7994.4  &  6936 & 6968.4 & 6918 & 6950.4  \\
7368 & 7400.4 &	7926 & 7958.4  &  7692 & 7724.4 & 7890 & 7922.4  \\
7080 & 7112.4 &	7044 & 7076.4  &  7134 & 7166.4 & 7548 & 7580.4  \\
7044 & 7076.4 &	7980 & 8012.4  &  7494 & 7526.4 & 7998 & 8030.4  \\
\hline
\end{tabular}
\caption {Feature and Continuum bandpasses selected for predicting $Metallicity$ 
      using noiseless BT\_Settl spectra with signal-to-noise ratios
      equal to $10$ and 50.} \label{tab:tab_SNR1050_G}
\end{center}
\end{table}

